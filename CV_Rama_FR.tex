%% start of file `template.tex'.
%% Copyright 2006-2010 Xavier Danaux (xdanaux@gmail.com).
%
% This work may be distributed and/or modified under the
% conditions of the LaTeX Project Public License version 1.3c,
% available at http://www.latex-project.org/lppl/.


\documentclass[11pt,a4paper]{moderncv}



% moderncv themes
%\moderncvtheme[blue]{casual}                 % optional argument are 'blue' (default), 'orange', 'red', 'green', 'grey' and 'roman' (for roman fonts, instead of sans serif fonts)
\moderncvtheme[orange]{classic}                % idem

% character encoding
\usepackage[utf8]{inputenc}                   % replace by the encoding you are using

% adjust the page margins
\usepackage[scale=0.9]{geometry}
\setlength{\hintscolumnwidth}{3cm}						% if you want to change the width of the column with the dates
%\AtBeginDocument{\setlength{\maketitlenamewidth}{6cm}}  % only for the classic theme, if you want to change the width of your name placeholder (to leave more space for your address details
\AtBeginDocument{\recomputelengths}                     % required when changes are made to page layout lengths

% personal data
%\photo[64pt]{picture.png}                         % '64pt' is the height the picture must be resized to and 'picture' is the name of the picture file; optional, remove the line if not wanted

\firstname{Rama}
\familyname{Herbin}
%\title{}               % optional, remove the line if not wanted


%\address{21 Rue Diderot}{38000 GRENOBLE}    % optional, remove the line if not wanted
\mobile{06 52 72 44 19}                    % optional, remove the line if not wanted
%\phone{phone (optional)}                      % optional, remove the line if not wanted
%\fax{fax (optional)}                          % optional, remove the line if not wanted
\email{rama.herbin@gmail.com}                      % optional, remove the line if not wanted
\phone{Né le 4 novembre 2000}                % optional, remove the line if not wanted
\homepage{www.rama.app}                % optional, remove the line if not wanted

%\quote{}                % optional, remove the line if not wanted

% to show numerical labels in the bibliography; only useful if you make citations in your resume
%\makeatletter
%\renewcommand*{\bibliographyitemlabel}{\@biblabel{\arabic{enumiv}}}
%\makeatother

% bibliography with mutiple entries
%\usepackage{multibib}
%\newcites{book,misc}{{Books},{Others}}

%\nopagenumbers{}                             % uncomment to suppress automatic page numbering for CVs longer than one page
%----------------------------------------------------------------------------------
%            content
%----------------------------------------------------------------------------------
\begin{document}
\maketitle

\section{Formation}

\cventry{2022 - 2024}{Gobelins}{$\:$Design \& Management de l'innovation interactive}{Paris}{}{Master Expert en Création Numérique Interactive}

\cventry{2020 - 2021}{Licence Pro Dev web}{$\:$Applications web, e-commerce et big data}{Aix-en-Provence}{}{}

\cventry{2018 - 2020}{DUT Informatique}{IUT de Valence}{Valence}{}{}

\cventry{2018}{Baccalauréat}{Lycée Armorin}{Crest}{}{}
%\vspace{0.5cm}


\section{Expériences}
\cventry{Septembre 2022 - Maintenant}{Développeur interactif}{Fleur de Papier}{Alternance}{\textit{Développement d'applications créatives sur Nuxt 3 avec Directus CMS}} {}
\cventry{Septembre 2021 - Septembre 2022}{Web developer}{Fleur de Papier}{CDD}{\textit{Développement d'applications interactives sur un framework MVC single-page / ou avec Vue.js {\linebreak \textbf{ Projets notables :}}} } {\begin{itemize}
	\item Jeux pédagogiques pour l'Atrium à Rouen
	\item Dispositifs interactifs pour la Bibliothèque nationale de France (site Richelieu)
\end{itemize}} {}
\cventry{Septembre 2020 - Septembre 2021}{Développeur web Junior}{Fleur de Papier}{Alternance}{\textit{Développement d'applications de médiation culturelle en Javascript sur un framework MVC single-page} {\linebreak \textbf{ Projet principal :}}} {\begin{itemize}
	\item Développement des interfaces des bornes du parcours muséographique de La Contemporaine
\end{itemize}}  {}
\cventry{Mai 2020 - Juillet 2020}{Stage}{Fleur de Papier}{Eurre}{\textit{Intégration web, création d'applications pour des usages internes}}{}
\cventry{Mars 2020 - Avril 2020}{Projet}{IUT d'Aix}{Aix-en-Provence}{\textit{Création d'un site web dynamique avec le framework Django (CRUD, Authentification, Session...)}}{}

\cventry{Septembre 2019 - Juin 2020}{Projet}{IUT de Valence}{Valence}{\textit{Création d'une borne interactive}}{Borne permettant aux étudiants de se renseigner sur les entreprises locales. Utilisation d'un Raspberry Pi, OpenStreetMap (uMap), et technologies web.}



\section{Compétences}
\subsection{Développement}
\cvline{Front-end}{HTML/(S)CSS, Javascript (ES6), GSAP, Vue.js, Nuxt 3, WebGL/Three.js, React}
\cvline{Back-end}{DirectusCMS, Node.js, Django, PostgreSQL, MongoDB }
\cvline{Méthodologie}{Design patterns (MVC, Singleton...), Clean Code, Merise, UML, Agile}
\cvline{Software/Tools}{VSCode, Git, Vim, Adobe Creative Suite, Figma, Virtual Box, Docker, Portainer, Vercel, Unreal Engine}
\cvcomputer{Environnement}{Linux, MacOs, Windows}{}{}

\subsection{Langues}
\cvlanguage{Français}{Maternelle}{}
\cvlanguage{Anglais}{C1}{}


\section{Centres d'intêret}
\cvline{Sports}{\small Musculation}
\cvline{Musique}{\small Guitare}
\cvline{Culture}{\small Films (VOST) et photographie}

\AtBeginShipout{\AtBeginShipoutUpperLeft{%
  \put(\paperwidth-2.8cm\relax,-29.4cm){{\tiny Made with LaTeX}}%
}}

\end{document}
%% end of file `template_en.tex'.
